% this TeX file provides an awesome example of how TeX will make super 
% awesome tables, at the cost of your of what happens when you try to make a
% table that is very complicated.
% Originally turned in for Dr. Nico's Security Class
\documentclass[11pt]{article}



% Use wide margins, but not quite so wide as fullpage.sty
\marginparwidth 0.5in 
\oddsidemargin 0.25in 
\evensidemargin 0.25in 
\marginparsep 0.25in
\topmargin 0.25in 
\textwidth 6in \textheight 8 in
% That's about enough definitions

% multirow allows you to combine rows in columns
\usepackage{multirow}
% tabularx allows manual tweaking of column width
\usepackage{tabularx}
% longtable does better format for tables that span pages
\usepackage{longtable}
\usepackage{graphicx}
\usepackage{placeins}
\usepackage[font=small,skip=0pt]{caption}

\begin{document}
% this is an alternate method of creating a title
%\hfill\vbox{\hbox{Gius, Mark}
%       \hbox{Cpe 456, Section 01}  
%       \hbox{Lab 1}    
%       \hbox{\today}}\par
%
%\bigskip
%\centerline{\Large\bf Lab 1: Security Audit}\par
%\bigskip
\author{Clindo Devassy K, Subhadeep Manna}
\title{DSP Lab 2016-17, JNI using Intel-SGX}
\maketitle
\tableofcontents

\section{Abstract}
\Abstract{Software applications frequently need to work with private information. The security of these information is crucial. Intel have introduced the Intel� Software Guard Extensions (Intel� SGX). A method to secure our code and data from disclosure. Intel SGX could be coded using C/C++. This project creates a JNI implementation for SGX calls and with those implementation creates an arithmetic evaluator. This arithmetic evaluator can be accessed remotely.\\\\\\\\\\\\}


\section{Application Flow}

 \begin{figure}[htb]
  \begin{center}
  \includegraphics[width=\linewidth]{FlowDiagram}
  \caption{Flow Diagram}
  \end{center}
  \end{figure}
\FloatBarrier

\section{Prerequisites}
\begin{enumerate}
  \item JDK 1.8 is installed on both client and server 
  \item The Server should be a SGX enabled machine.
\end{enumerate}
\section{SGX Server Usage}
\begin{enumerate}
  \item Copy the server side project for SGX
  \item Run 'make'
   \item export LD\_LIBRARY\_PATH='/opt/intel/sgxsdk/sdk\_libs/'
   \item  javac -cp . JavaApp.java
   \item  java -Djava.library.path=. JavaApp
\end{enumerate}

\section{Java Client-Server Usage}

\subsection{Server}
\begin{enumerate}
\item Copy the file Server.java
\item Compile it using javac Server.java
\item Run the server with java Server
\end{enumerate}

\subsection{Client}
\begin{enumerate}
\item Copy Client.java and EncryptionUtil.java to same folder
\item Compile client using javac Client.java
\item Run the server with java Client
\end{enumerate}
\Java{Note: We have save common erros and its solutions inside the Errors and Troubleshooting directory in the repository}
\section{Pending tasks}
\begin{enumerate}
\item Integrating the java client-server with SGX server
\item Remote Attestation of the server
\item Exception Handling in Client-Server Communication
\end{enumerate}

\section{Questions}
\begin{enumerate}
\item Should the public key received by the server be saved in side the enclave or outside the enclave?
\item Should we transfer the computed result using OCALL outside the enclave? Currently we are just making a test OCALL and printing it directly.

\end{enumerate}

\end{document}
